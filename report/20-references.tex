\begin{center}
    \MakeUppercase{\large Список использованных источников}
\end{center}
\addcontentsline{toc}{section}{СПИСОК ИСПОЛЬЗОВАННЫХ ИСТОЧНИКОВ} % Добавляем в оглавление

\renewcommand{\refname}{} % Удаляем "Список литературы"
\vspace{-11mm}

\makeatletter
\renewcommand\@biblabel[1]{#1.}
\makeatother

\begin{thebibliography}{9}

    \bibitem{saruev_rudachenko} Саруев А. Л., Рудаченко А. В. \textit{ИССЛЕДОВАНИЯ НАПРЯЖЕННО-ДЕФОРМИРОВАННОГО СОСТОЯНИЯ ТРУБОПРОВОДОВ}~---~Издательство Томского политихнического университета, 2021, с. 5.
    
    \bibitem{komarov} Комаров В. И. \textit{Путеводитель по базам данных}~---~ДМК-Пресс, 2024, с. 21-58.
    
    \bibitem{vas_hus} Васильева К. Н., Хусаинова Г. Я. \textit{Реляционные базы данных}~---~СФ БашГУ, 2019, с. 22-23.
    
    \bibitem{luchinina} Лучинина З. С., Сидоркина И. Г. \textit{МАТЕМАТИЧЕСКАЯ МОДЕЛЬ ДОКУМЕНТНО-ОРИЕНТИРОВАННОЙ БАЗЫ ДАННЫХ}~---~Вестник Чувашского университета, 2015, с. 174-179.
    
    \bibitem{eldar} Эльдарханов А. М. \textit{Обзор моделей данных объектно-ориентированных СУБД}~---~Труды Института системного программирования РАН, 2011, с. 205-224.
    
    \bibitem{otradnov} Отраднов К. К., Алёшкин А. С., Калинин В. Н.\textit{ПРЕИМУЩЕСТВА ИСПОЛЬЗОВАНИЯ ГРАФОВЫХ БАЗ ДАННЫХ ПРИ РАЗРАБОТКЕ ПРИКЛАДНЫХ ПРИЛОЖЕНИЙ}~---~Вестник РГРТУ, 2023, с. 73-83.
    
\end{thebibliography}
