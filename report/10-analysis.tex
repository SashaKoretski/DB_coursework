% Основная часть
\section{Аналитическая часть}

\noindent
\hspace{0.75cm}
В данном разделе проведена формализация задачи, формализация данных, анализ баз данных и систем управления базами данных, подходящих для данной задачи.

\subsection{Формализация задачи}

\noindent
\hspace{0.75cm}
Дана трубка, представленная набором сечений, каждое из которых состоит из точек $P_i = {p_{i1}, p_{i2}, ..., p_{in_i}}$, соединенных с несколькими другими точками, в том числе из соседних сечений. Каждая точка $p_{ij}$ имеет координаты $(x_{ij}, y_{ij}, z_{ij})$.

\noindent
\hspace{0.75cm}
Пользователь выбирает точку, относительно которой будет происходить деформация и направление деформации. Во время деформации трубки, история сечений будет сохраняться в базу данных. Деформация будет происходить по следующему закону: (здесь будет формализация модели деформации).

\subsection{Формализация данных}

\noindent
\hspace{0.75cm}
Основываясь на сформулированных требованиях, база данных должна содержать информацию об:

\begin{itemize}
	\item изначальной модели трубки;
	\item деформируемой трубке;
	\item историю деформации трубки;
	\item сечениях, содержащихся в трубке;
	\item точках, составляющих сечение;
	\item ребрах, соединяющих пары точек;
	\item пользователях.
\end{itemize}

\noindent
\hspace{0.75cm}
Связи между описанными сущностями представлены на ER-диаграмме (рисунок).