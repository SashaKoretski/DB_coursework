\section{Конструкторская часть}

\noindent
\hspace{0.75cm}
В этом разделе рассматриваются основные алгоритмы, предназначенные для вычисления расстояний Левенштейна и Дамерау~---~Левенштейна с помощью рекуррентных формул из прошлого раздела.

\subsection{Алгоритмы вычисления расстояния Левенштейна}

\noindent
\hspace{0.75cm}
Алгоритмы, рассчитывающие расстояние Левенштейна, можно реализовать с использованием итеративных и рекурсивных методов. В частности, метод заполнения матрицы расстояний позволяет последовательно находить минимальное количество операций для приведения одной строки к другой. Блок-схема такого итеративного алгоритма представлена на рисунке~\ref{fig:l_matrix_design}.

\begin{figure}[H]
    \centering{\includegraphics[scale=0.7]{img/l_matrix.png}}
    \caption{Блок-схема итеративного алгоритма вычисления расстояния Левенштейна}
    \label{fig:l_matrix_design}
\end{figure}

\noindent
\hspace{0.75cm}
Рекурсивная версия алгоритма Левенштейна без кеширования, изображенная на рисунке~\ref{fig:l_recursion_basic}, использует прямой рекурсивный вызов для нахождения расстояния.

\begin{figure}[H]
    \centering{\includegraphics[scale=0.8]{img/l_recursion_basic.png}}
    \caption{Блок-схема рекурсивного алгоритма Левенштейна без кеширования}
    \label{fig:l_recursion_basic}
\end{figure}

\noindent
\hspace{0.75cm}
Алгоритм, использующий рекурсивный подход с кешированием, как показано на рисунке~\ref{fig:l_recursion_optimized}, оптимизирует вычисления, сохраняя промежуточные значения.

\begin{figure}[H]
    \centering{\includegraphics[scale=0.8]{img/l_recursion_optimized.png}}
    \caption{Блок-схема рекурсивного алгоритма Левенштейна с кешированием}
    \label{fig:l_recursion_optimized}
\end{figure}

\subsection{Алгоритмы вычисления расстояния Дамерау~---~Левенштейна}

\noindent
\hspace{0.75cm}
Алгоритм Дамерау~---~Левенштейна представляет собой расширение метода Левенштейна, добавляя к базовым операциям замену соседних символов. Рекурсивный алгоритм без кеширования, представленный на рисунке~\ref{fig:dl_recursion_basic}, позволяет учитывать дополнительные ошибки, вызванные перестановкой букв.

\begin{figure}[H]
    \centering{\includegraphics[scale=0.8]{img/dl_recursion_basic.png}}
    \caption{Блок-схема рекурсивного алгоритма Дамерау~---~Левенштейна без кеширования}
    \label{fig:dl_recursion_basic}
\end{figure}
