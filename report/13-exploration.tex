\section{Исследовательская часть}

\subsection{Пример работы}

\noindent
\hspace{0.75cm}
На рисунке~\ref{img:work_example} представлен пример работы программы.

\begin{figure}[H]
     \centering{\includegraphics[scale=0.5]{img/work_example.jpg}}
    \caption{Пример работы программы}
    \label{img:work_example}
\end{figure}

\subsection{Технические характеристики}

\noindent
\hspace{0.75cm}
Тестирование программы проводилось на устройстве с операционной системой Ubuntu 22.04.5, оснащённом процессором Intel Core i5---4210U и оперативной памятью объёмом 8 Гб с помощью Arm Cortex-M7(STM32F767 Nucleo-144) с оперативной памятью объёмом 512 Кб. Ноутбук был подключён к сети питания и не использовался для других задач во время тестирования.

\subsection{Время выполнения реализаций алгоритмов}

\noindent
\hspace{0.75cm}
Все реализации алгоритмов были протестированы на случайно сгенерированных строках длиной от 0 до 9. Каждое значение замерялось 100 раз для устранения погрешности, после чего вычислялось среднее время работы.

\begin{figure}[H]
    \centering
    \includegraphics[width=0.8\textwidth]{img/time_all.png}
    \caption{Сравнение времени работы всех реализаций алгоритмов}
    \label{img:time_all}
\end{figure}

\subsection{Таблица временных характеристик}

\noindent
\hspace{0.75cm}
Средние значения времени выполнения алгоритмов представлены в таблице~\ref{tab:times_divided}. Результаты демонстрируют зависимости времени работы алгоритмов от длины строк.

\begin{table}[h!]
    \caption{Время работы алгоритмов (мкс)}
    \begin{tabular}{|c|S[table-format=2.4]|S[table-format=2.4]|S[table-format=3.4]|S[table-format=3.4]|}
        \hline
        \textbf{Длина} & \textbf{Л.(итер.)} & \textbf{Л.(рек. с кешем)} & \textbf{Л.(рек. без кеша)} & \textbf{Д.-Л.(рек. без кеша)} \\ \hline
        0 & 1.0027 & 1.2628 & 0.3021 & 0.2923 \\ \hline
        1 & 1.5858 & 2.0406 & 0.8796 & 0.8637 \\ \hline
        2 & 2.9373 & 4.4822 & 3.5807 & 3.7993 \\ \hline
        3 & 6.2976 & 10.1573 & 22.0523 & 23.3493 \\ \hline
        4 & 7.0305 & 11.8198 & 80.1471 & 89.3529 \\ \hline
        5 & 10.5013 & 17.6816 & 442.3224 & 483.6511 \\ \hline
        6 & 11.1186 & 19.2405 & 1851.4005 & 2039.1925 \\ \hline
        7 & 16.3186 & 26.5480 & 10638.5296 & 11692.0248 \\ \hline
        8 & 29.7772 & 52.3268 & 86549.5974 & 94950.0442 \\ \hline
        9 & 35.6784 & 61.4473 & 446796.3284 & 489296.0056 \\ \hline
    \end{tabular}
    \label{tab:times_divided}
\end{table}

\subsection{Сравнение реализаций}

\noindent
\hspace{0.75cm}
На рисунке~\ref{img:time_with_matrix} приведено сравнение времени выполнения итерационного и рекурсивного (с кешированием) алгоритмов Левенштейна.

\begin{figure}[H]
    \centering
    \includegraphics[width=0.8\textwidth]{img/time_with_matrix.png}
    \caption{Сравнение времени работы реализаций алгоритмов, использующих матрицы}
    \label{img:time_with_matrix}
\end{figure}

\noindent
\hspace{0.75cm}
На рисунке~\ref{img:time_rec_without_cash} показано сравнение времени работы рекурсивных реализаций алгоритмов Левенштейна и Дамерау~---~Левенштейна без кеширования.

\begin{figure}[H]
    \centering
    \includegraphics[width=0.8\textwidth]{img/time_rec_without_cash.png}
    \caption{Сравнение времени работы рекурсивных реализаций алгоритмов без кеширования}
    \label{img:time_rec_without_cash}
\end{figure}

\subsection*{Вывод}

\noindent  
\hspace{0.75cm} 
Рекурсивные алгоритмы выигрывают по памяти у итерационного алгоритма, так как максимальное использование памяти в первых растёт пропорционально сумме длин строк, тогда как в итерационном алгоритме — пропорционально их произведению. 

\noindent  
\hspace{0.75cm}  
Рекурсивные алгоритмы Левенштейна без кеша и Дамерау~---~Левенштейна показывают значительно более длительное время выполнения по сравнению с двумя другими реализациями. На длине строк 5 они работают в \( \frac{114.69468}{6.91644} \approx 16.6 \) и \( \frac{127.88060}{7.47511} \approx 17.1 \) раз медленнее итерационного алгоритма соответственно. На длине строк 9 это различие становится ещё более существенным: \( \frac{446796.327}{35.6784} \approx 12522.5 \) для Левенштейна без кеша и \( \frac{489296.568}{39.9677} \approx 12244.2 \) для Дамерау~---~Левенштейна.  

\noindent  
\hspace{0.75cm} 
Для первой пары алгоритмов (Левенштейн без кеша и Дамерау~---~Левенштейна) графики демонстрируют практически идентичное поведение, что позволяет выбирать любой из них без существенной разницы в производительности.

\noindent  
\hspace{0.75cm} 
Во второй паре алгоритмов (итерационный и рекурсивный с кешем) итерационный подход показывает преимущество по времени выполнения. Тем не менее, рекурсивный алгоритм с кешем может быть предпочтительным в задачах, где ограничение по памяти важнее скорости работы. 

