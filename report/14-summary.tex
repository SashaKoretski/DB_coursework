\begin{center}
    \MakeUppercase{\large Заключение}
\end{center}
\addcontentsline{toc}{section}{ЗАКЛЮЧЕНИЕ}

\noindent
\hspace{0.75cm}
В результате выполнения лабораторной работы при исследовании алгоритмов нахождения расстояний Левенштейна и Дамерау~---~Левенштейна были применены и отработаны навыки динамического программирования.

\noindent
\hspace{0.75cm}
В ходе выполнения лабораторной работы были выполнены следующие задачи:
\renewcommand{\labelenumi}{\theenumi}
\begin{enumerate}
  \item рассмотрены алгоритмы вычисления расстояния Левенштейна и Дамерау~---~Левенштейна;
  \item разработаны и проанализированы алгоритмы поиска этих расстояний;
  \item проведён сравнительный временной анализ алгоритмов вычисления расстояния Левенштейна и Дамерау~---~Левенштейна;
  \item описаны и обоснованы полученные результаты в отчёте о выполненной лабораторной работе.
\end{enumerate}

\noindent
\hspace{0.75cm}
В рамках лабораторной работы были рассмотрены, спроектированы и реализованы алгоритмы для вычисления расстояний Левенштейна и Дамерау~---~Левенштейна в их итеративной, рекурсивной и комбинированной версиях. Для проверки корректности работы всех реализаций были разработаны тесты, охватывающие пограничные случаи, результаты которых соответствуют ожиданиям.

\noindent
\hspace{0.75cm}
Анализ разработанных программ продемонстрировал, что рекурсивные алгоритмы выигрывают по памяти у итерационных, так как их пиковое использование памяти растёт пропорционально сумме длин строк, в то время как у итерационных~---~пропорционально их произведению.

\noindent
\hspace{0.75cm}
Рекурсивные реализации алгоритма Левенштейна без кеша и алгоритма Дамерау~---~Левенштейна при работе со строками длиной 5 требуют в 17 раз больше времени чем итерационные подходы, а при длине строки 9 — более чем в 12 тысяч раз. Алгоритмы из первой пары (рекурсивные без кеша) демонстрируют схожие результаты, позволяя выбрать любой из них в зависимости от задачи. Во второй паре (итерационный и рекурсивный с кешем) итерационный алгоритм показывает лучшее время выполнения, однако рекурсивный с кешем может быть предпочтительным при ограничениях на использование памяти.
