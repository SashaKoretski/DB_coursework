\begin{center}
    \MakeUppercase{\large ЗАКЛЮЧЕНИЕ}
\end{center}
\addcontentsline{toc}{section}{ЗАКЛЮЧЕНИЕ}

\noindent
\hspace{1.25cm}
В рамках данной работы была:

\begin{enumerate}

\item определена информационная модель предметной области;

\item спроектирована структура базы данных и установлены ограничения целостности данных, обеспечивающие корректность хранения геометрической информации;

\item выбрана система управления базами данных;

\item разработан интерфейс доступа к базе данных, позволяющий выполнять основные операции с геометрическими объектами;

\item исследована зависимость времени выполнения операции деформации от количества сечений;

\item исследована зависимость времени выполнения операции деформации от количества точек в сечении.


\end{enumerate}

\noindent
\hspace{1.25cm}
Проведенные исследования показали, что время построения деформированной трубки возрастает линейно как с увеличением количества сечений, так и с ростом числа точек в каждом сечении. Это свидетельствует о линейной зависимости вычислительной сложности от объема входных данных.