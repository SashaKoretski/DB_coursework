\begin{center}
    \MakeUppercase{\large ВВЕДЕНИЕ}
\end{center}
\addcontentsline{toc}{section}{ВВЕДЕНИЕ}

\noindent
\hspace{1.25cm}
Тонкостенные трубчатые поверхности используются при моделировании объектов в системах автоматизированного проектирования, медицинской визуализации и промышленном дизайне. В процессе их построения возникает задача плавной интерполяции между заданными сечениями после деформации трубки, обеспечивающей непрерывность и гладкость полученной поверхности~\cite{Ivanov}.

\noindent
\hspace{1.25cm}
Целью работы является разработка базы данных для хранения и управления историей изменения поперечных сечений трубки после деформации.

\noindent
\hspace{1.25cm}
Для достижения поставленной цели необходимо решить следующие задачи:

\renewcommand{\labelenumi}{\theenumi}
\begin{enumerate}

\item определить информационную модель предметной области и выделить ключевые сущности;

\item спроектировать структуру базы данных и установить ограничения целостности данных;

\item выбрать СУБД для хранения геометрических данных трубки;

\item разработать интерфейс доступа к базе данных;

\item исследовать зависимость времени выполнения операции деформации от количества сечений;

\item исследовать зависимость времени выполнения операции деформации от количества точек в сечении.

\end{enumerate}

\newpage